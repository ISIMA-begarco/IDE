Introduction

	Il existe aujourd'hui une multitude de façons de produire du code et par extension un programme. Elles sont plus ou moins conseillées selon le support, le langage de programmation et surtout l'utilisateur. De façon générale, tout ce qui nous est proposé reflète une qualité certaine et un avantage non négligeable pour le programmeur si toutefois les bonnes pratiques de programmation sont respectées.
	C'est donc pour facilité la vie des programmeurs et surtout pour pallier leur oisiveté qu'ont été créés les Environnements de Développement Intégré ( EDI ) ou encore integrated development environment ( IDE ) en anglais. En effet, au début de l'ère informatique, le développement ne nécessitait pas autant de fichiers et autres car l'on utilisait des cartes perforées. C'est donc avec l'apparition de la programmation générique actuelle que sont apparus les EDI. Le premier étant Dartmouth BASIC en 1964.
	Un EDI est donc un programme regroupant de multiples sous programmes, facilitant la production et la mise en œuvre de code. Mais alors, comment un IDE peut-il assister un programmeur ?
	Nous verrons donc qu'il existe tout une liste de fonctionnalités qui font d'un IDE une solution de développement intéressante puis nous concentrerons notre attention sur l'exemple de CodeBlocks.




I – IDE : un éventail de fonctionnalités en un programme

II – CodeBlocks : un exemple d'application
