\documentclass[a4paper, 12pt]{article} 

\usepackage [frenchb]{babel} 
\usepackage{color} 
\usepackage[utf8]{inputenc}
\usepackage[top=3cm, bottom=4cm, left=3cm, right=3cm]{geometry}
\usepackage{setspace}

\title{\underline{\textbf{IDE : Integrated Development Environment}}} 
\author{Benoît GARCON, Loïck CHIRON, Marie CASSAING} 
\date{11/04/2013}

\begin{document}

\definecolor{bleuciel}{RGB}{168,248,248} 
\definecolor{blanc}{RGB}{255,255,255} 
\pagecolor{bleuciel}

\maketitle \vspace {4cm} \thispagestyle{empty}

\setcounter{page}{0}

\newpage \renewcommand{\contentsname}{Sommaire} \tableofcontents
\pagecolor{blanc}

\newpage \section*{\underline{Introduction}} \addcontentsline{toc}{section}{Introduction}

\begin{doublespace}


	Il existe aujourd'hui une multitude de façons de produire du code et par extension un programme. Elles sont plus ou moins conseillées selon le support, le langage de programmation et surtout l'utilisateur. De façon générale, tout ce qui nous est proposé reflète une qualité certaine et un avantage non négligeable pour le programmeur si toutefois les bonnes pratiques de programmation sont respectées.

	C'est donc pour facilité la vie des programmeurs et surtout pour pallier leur oisiveté qu'ont été créés les Environnements de Développement Intégré ( EDI ) ou encore integrated development environment ( IDE ) en anglais. En effet, au début de l'ère informatique, le développement ne nécessitait pas autant de fichiers et autres car l'on utilisait des cartes perforées. C'est donc avec l'apparition de la programmation générique actuelle que sont apparus les EDI. Le premier étant Dartmouth BASIC en 1964.

	Un EDI est donc un programme regroupant de multiples sous programmes, facilitant la production et la mise en œuvre de code. Mais alors, comment un IDE peut-il assister un programmeur ?

	Nous verrons donc qu'il existe tout une liste de fonctionnalités qui font d'un IDE une solution de développement intéressante puis nous concentrerons notre attention sur l'exemple de CodeBlocks.
\end{doublespace}

\newpage \section{\underline{IDE : un éventail de fonctionnalités en un programme}}

\begin{doublespace}
Un environnement de développement comporte en général tous les outils nécessaires pour analyser, écrire et debugger votre programme. On retrouvera donc dans tous les IDE, un éditeur de texte incorporé comportant une syntaxe typique de mise en valeur de votre code par l'intermédiaire d'une indentation automatique, de mise en couleur des mots clés et d'une complétion automatique de ces mots clés. Il y aura également dans la plupart des cas un débuggeur et un compilateur de manière à pouvoir rédiger, corriger et tester votre programme avec un seul logiciel. De manière générale, le processus de débogage commence pendant la génération. La génération de votre programme  aide à détecter les erreurs de compilation et vous affiche les numéros de lignes correspondants. Les erreurs les plus fréquentes étant dues à une syntaxe incorrecte, à des mots clés mal orthographiés ou à une incompatibilité de type.  Enfin, un IDE est également très souvent muni d’un système de gestion de projet très pratique lorsque l’on va réaliser un programme de manière collaboratif en sein d’une entreprise par exemple. Cela regroupe donc  l’interconnexion des données, dossiers et fichiers nécessaire à la réalisation du projet.
\end{doublespace}

\newpage \section{\underline{CodeBlocks : un exemple d'application}}

\begin{doublespace}
\subsection{\underline{Un premier aperçu}} 
	Nous avons choisis ici de vous présenter un IDE particulier puisqu'il est connu par la majorité des membres du groupe : il s'agit Code::Blocks qui fait parti des IDE les plus répandus pour les langages C et C++. Voici donc un outil privilégié pour les développeurs en herbe et peut-être même les plus confirmés. Les raisons d'un tel succès sont assez simple :

- libre et " open source " : c'est un logiciel gratuit donc accessible à tous mais aussi perfectible par tous

- multiplateforme : que l'on développe sous Linux, Mac OS X ou même Windows il existe une version disponible

- complet : Code::Blocks offre une multitude de fonctionnalités pratiques que l'on peut attendre d'un bon IDE, mais s'il manque un outil il y aura certainement un plug-in pour pallier ce défaut.

	L'aventure Code::Blocks commença en 2005, il a été développé en C++ grâce à la bibliothèque wxWidgets, ce qui explique pourquoi il est si simple d'améliorer cet IDE. Code::Blocks est un logiciel assez intuitif et simple d'utilisation (mais il reste toutefois très complet). Bien qu'utilisable en ligne de commande, il arbore aussi une interface graphique très classique. On retrouve en haut de la fenêtre les différents menus et boutons donnant accès à toutes les fonctionnalités.  Le reste est divisé en deux zones : 

la zone centrale étant composé de l'indispensable éditeur de texte pour taper le code associé à un explorateur de fichiers

la console de sortie au bas de l'écran regroupe les différents résultats des opérations effectuées et les classe par catégories : recherche, debugger, compilateur …

Nous allons maintenant voir les différentes étapes pour concevoir un programme sous Code::Blocks.

\subsection{\underline{La création d'un projet}} 

\subsection{\underline{La programmation}} 

\subsection{\underline{La compilation}} 
\end{doublespace}

\newpage \section*{\underline{Conclusion}} \addcontentsline{toc}{section}{Conclusion} 

\begin{doublespace}
BLABLA BLA BLABLA BLA BLABLA BLA BLABLA BLA BLABLA BLA BLABLA BLA BLABLA BLA BLABLA BLA BLABLA BLA BLABLA BLA BLABLA BLA BLABLA BLA BLABLA BLA BLABLA BLA BLABLA BLA BLABLA BLA BLABLA BLA BLABLA BLA BLABLA BLA BLABLA BLA BLABLA BLA BLABLA BLA BLABLA BLA BLABLA BLA BLABLA BLA BLABLA BLA BLABLA BLA BLABLA BLA BLABLA BLA BLABLA BLA BLABLA BLA BLABLA BLA BLABLA BLA BLABLA BLA BLABLA BLA BLABLA BLA BLABLA BLA BLABLA BLA BLABLA BLA BLABLA BLA BLABLA BLA BLABLA BLA BLABLA BLA BLABLA BLA BLABLA BLA BLABLA BLA BLABLA BLA BLABLA BLA BLABLA BLA BLABLA BLA BLABLA BLA BLABLA BLA BLABLA BLA BLABLA BLA BLABLA BLA BLABLA BLA BLABLA BLA BLABLA BLA BLABLA BLA BLABLA BLA BLABLA BLA BLABLA BLA BLABLA BLA BLABLA BLA BLABLA BLA BLABLA BLA BLABLA BLA BLABLA BLA BLABLA BLA BLABLA BLA BLABLA BLA BLABLA BLA BLABLA BLA BLABLA BLA BLABLA BLA BLABLA BLA BLABLA BLA BLABLA BLA BLABLA BLA BLABLA BLA BLABLA BLA BLABLA BLA BLABLA BLA BLABLA BLA BLABLA BLA BLABLA BLA BLABLA BLA BLABLA BLA BLABLA BLA BLABLA BLA BLABLA BLA BLABLA BLA BLABLA BLA BLABLA BLA BLABLA BLA BLABLA BLA BLABLA BLA BLABLA BLA BLABLA BLA BLABLA BLA BLABLA BLA BLABLA BLA BLABLA BLA BLABLA BLA BLABLA BLA BLABLA BLA BLABLA BLA BLABLA BLA BLABLA BLA BLABLA BLA BLABLA BLA BLABLA BLA BLABLA BLA BLABLA BLA BLABLA BLA BLABLA BLA BLABLA BLA BLABLA BLA BLABLA BLA BLABLA BLA BLABLA BLA BLABLA BLA BLABLA BLA BLABLA BLA BLABLA BLA BLABLA BLA BLABLA BLA BLABLA BLA BLABLA BLA BLABLA BLA BLABLA BLA BLABLA BLA BLABLA BLA BLABLA BLA BLABLA BLA BLABLA BLA BLABLA BLA BLABLA BLA BLABLA BLA BLABLA BLA BLABLA BLA BLABLA BLA BLABLA BLA BLABLA BLA BLABLA BLA BLABLA BLA BLABLA BLA BLABLA BLA BLABLA BLA BLABLA BLA BLABLA BLA BLABLA BLA BLABLA BLA BLABLA BLA BLABLA BLA BLABLA BLA BLABLA BLA BLABLA BLA BLABLA BLA BLABLA BLA BLABLA BLA BLABLA BLA BLABLA BLA BLABLA BLA BLABLA BLA BLABLA BLA BLABLA BLA BLABLA BLA BLABLA BLA BLABLA BLA BLABLA BLA BLABLA BLA BLABLA BLA BLABLA BLA BLABLA BLA BLABLA BLA BLABLA BLA BLABLA BLA BLABLA BLA BLABLA BLA BLABLA BLA.
\end{doublespace}

\end{document}
